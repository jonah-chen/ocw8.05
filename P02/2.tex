\begin{sol}
\begin{enumerate}[label=\textbf{(\alph*)}]
\item
The left hand side of the identity can be rewritten as
\begin{equation}
\begin{aligned}
    (\psi_{k+1}\psi_k'-\psi_k\psi_{k+1}')|_a^b&=\int_a^b \frac{d}{dx}(\psi_{k+1}\psi_k'-\psi_k\psi_{k+1}')dx\\
    &=\int_a^b\psi_{k+1}'\psi_k'+\psi_{k+1}\psi_k''-\psi_k'\psi_{k+1}'-\psi_k\psi_{k+1}''dx
    =\int_a^b\psi_{k+1}\psi_k''-\psi_k\psi_{k+1}''dx
\end{aligned}
\end{equation}
Replacing $\psi''$ with $(U(x)-\epsilon)\psi$
\begin{equation}
\begin{aligned}
    &=\int_a^b\psi_{k+1}\psi_k(U(x)-\epsilon_k)-\psi_k\psi_{k+1}(U(x)-\epsilon_{k+1})dx\\
    &=\int_a^b\psi_k\psi_{k+1}(\epsilon_{k+1}-\epsilon_k)dx=(\epsilon_{k+1}-\epsilon_k)\int_a^b\psi_k\psi_{k+1}dx
\end{aligned}
\end{equation}
\item
If $\psi_{k}(a)=\psi_{k}(b)=0, \epsilon_{k+1}>\epsilon_k,\psi_{k}>0\forall x\in(a,b)\implies\exists x_0\in(a,b):\psi_{k+1}(x_0)=0$.
\begin{proof}
Assume $\exists\psi_{k+1}: sgn(\psi_{k+1})=C$  for $x\in(a,b)$, where $C$ is a non-zero constant. \\

\begin{align}
    (\epsilon_{k+1}-\epsilon_k)\int_a^b\psi_k\psi_{k+1}dx&=(\psi_{k+1}\psi_k'-\psi_k\psi_{k+1}')|_a^b=\psi_{k+1}\psi_k'|_a^b\\
    \int_a^b\psi_k\psi_{k+1}dx&=\frac{\psi_{k+1}\psi_k'|_a^b}{(\epsilon_{k+1}-\epsilon_k)}
\end{align}
Given that $\psi_k$ is continuous and $\psi_k(a)=\psi_k(b)=0, \psi_k(x)>0\forall x\in(a,b)$\\ 
$\implies\psi_k'(a)>0,\psi_k'(b)<0$.\\
Let $\alpha,\beta$ be positive real constants. The previous equation can be rewritten as
\begin{equation}
    \int_a^b\psi_k\psi_{k+1}dx=-(\alpha\psi_{k+1}(b)+\beta\psi_{k+1}(a))
\end{equation}
Firstly, consider the case where $sgn(\psi_{k+1})=1\forall x\in(a,b)$ $\therefore \psi_{k+1}(a)\geq 0,\psi_{k+1}(b)\geq 0$. Then,
\begin{equation}
    \int_a^b\psi_k\psi_{k+1}dx\leq 0
\end{equation}
Since it is defined that $\psi_{k}>0\forall x\in(a,b)$, the integral trivially evaluates to a positive number. This leads to a contradiction.\\
Secondly, consider the case where  $sgn(\psi_{k+1})=-1\forall x\in(a,b)$ $\therefore \psi_{k+1}(a)\leq 0,\psi_{k+1}(b)\leq 0$. \\Then,
\begin{equation}
    \int_a^b\psi_k\psi_{k+1}dx\geq 0
\end{equation}
Since it is defined that $\psi_{k}>0\forall x\in(a,b)$, the integral trivially evaluates to a negative number. This also leads to a contradiction.\\ It has been shown that $sgn(\psi_{k+1})$ cannot be constant in the interval $(a,b)$, and two points $x_1,x_2\in(a,b):\psi_{k+1}(x_1)>0,\psi_{k+1}(x_2)<0$ can be chosen. According to the intermediate value theorem, $\exists x_0\in(x_1,x_2):\psi_{k+1}(x_0)=0$. $\because(x_1,x_2)\subset(a,b)\exists x_0\in(a,b):\psi_{k+1}(x_0)=0$. QED.
\end{proof}
\end{enumerate}
\end{sol}