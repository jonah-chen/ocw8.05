\newpage
\begin{sol}
\begin{theorem}
An attractive potential in one dimension $V(x)$ is a potential where:
\begin{enumerate}
\item $V(x) \text{ is piece-wise continuous}$
\item $\lim_{x\to\pm\infty}V(x)=0$
\item $V(x)=-|V(x)|\forall x\in\mathbb{R}$
\item $V(x)\neq 0$ 
\end{enumerate}
$$\forall V(x)\implies \exists\psi(x):\left(-\frac{1}{2m}\frac{d^2}{dx^2}+V(x)\right)\psi=E\psi(x), E<0$$
\end{theorem}
\begin{lemma}
$\exists x_0:V(x_0)\equiv -2v_0<0$ 
\end{lemma}
\begin{proof}
Assume $\nexists x_0:V(x_0)<0$\\
$\because V(x)\neq 0, \exists x_1:V(x_1)>0$ \\
$\therefore V(x_1)=-|V(x_1)|=-V(x_1)$\\
$\because V(x_1)>0$, there is a contradiction.
\end{proof}
\begin{lemma}
$\exists x_0:V(x_0)\equiv -2v_0<0\implies\exists (x_1, x_2):x_0\in (x_1, x_2),V(x)\leq -v_0\forall x\in(x_1, x_2)$ 
\end{lemma}
\begin{proof}
I can't prove this just assume that it is true.
\end{proof}
\begin{lemma}
Use an interval $(x_1,x_2)$ from Lemma 2, where $V(x)\leq v_0$.
$$\bar{V}(x)\equiv v_0 \text{ for } x\in(x_1,x_2),\bar{V}(x)\equiv 0 \text{ elsewhere}$$
If
$$\hat{H}\equiv -\frac{1}{2m}\frac{d^2}{dx^2}+V(x)$$
$$\bar{H}\equiv -\frac{1}{2m}\frac{d^2}{dx^2}+\bar V(x)$$
Then
$$\bra\psi\hat H\ket\psi\leq\bra\psi\bar H\ket\psi\forall\psi$$
\end{lemma}
\begin{proof}
$$\bra\psi\hat O\ket\psi=\int_{-\infty}^\infty\psi^*\hat O\psi dx$$
$$\int_{-\infty}^\infty\psi^*\left(-\frac{1}{2m}\frac{d^2\psi}{dx^2}+V(x)\psi(x)\right)dx\leq\int_{-\infty}^\infty\psi^*\left(-\frac{1}{2m}\frac{d^2\psi}{dx^2}+\bar V(x)\psi(x)\right)dx$$
$$\int_{-\infty}^\infty V(x)|\psi(x)|^2dx\leq\int_{-\infty}^\infty\bar V(x)|\psi(x)|^2dx$$
Replacing the interval $(-\infty,\infty)$ with $(-\infty,x_1)$, $(x_1,x_2)$, and $(x_2, \infty)$ will yield:

$$\int_{-\infty}^{x_1} V(x)|\psi(x)|^2dx+\int_{x_1}^{x_2} V(x)|\psi(x)|^2dx+\int_{x_2}^\infty V(x)|\psi(x)|^2dx\leq \int_{x_1}^{x_2} -v_0|\psi(x)|^2dx$$ $$\because V(x)\leq -v_0, \int_{x_1}^{x_2} V(x)|\psi(x)|^2dx\leq \int_{x_1}^{x_2} -v_0|\psi(x)|^2dx$$
$$\therefore\int_{-\infty}^{x_1} V(x)|\psi(x)|^2dx+\int_{x_2}^\infty V(x)|\psi(x)|^2dx\leq 0$$
This inequality must be satisfied $\because V(x)\leq 0\forall x\in\mathbb{R}$. 
\end{proof}
\begin{lemma}
If 
\begin{enumerate}
\item $f(x)$ has a continuous derivative  $\forall x>a$
\item $\lim_{x\to a^+}f(x)=K$
\item $\lim_{x\to a^+}f'(x)=-\infty$
\end{enumerate}
$$\implies\exists x_0>a:f(x_0)<K$$
\end{lemma}
\begin{proof}
$\because f'(x)$ is continuous $\forall x>a$, $\exists\delta>0:f'(x)<-1\forall x\in(a,a+\delta)$\\
$\therefore f(x)$ satisfies the hypotheses of the \textit{Mean Value Theorem} on the interval $[a,a+\delta]$\\
Applying the \textit{Mean Value Theorem} yields
$$f'(c)=\frac{f(a+\delta)-f(a)}{\delta}<-1$$ $$f(a+\delta)-K<-\delta$$ Let $x_0=a+\delta$ 
$$f(x_0)<K-\delta<K$$
\end{proof}
\begin{proof}
Let a set of trial states be
$$\braket{x}{\psi_\alpha}=\Big(\frac{\alpha}{\pi}\Big)^\frac{1}{4}e^{-\frac{1}{2}\alpha x^2}$$
The ground state energy is bounded below $\bra\psi_\alpha\hat H\ket\psi_\alpha$. Using Lemma 3, the ground state energy is also bounded below $\bra\psi_\alpha\bar H\ket\psi_\alpha$.
$$\bra\psi_\alpha\bar H\ket\psi_\alpha=\int_{-\infty}^{\infty} \psi^*\left(-\frac{1}{2m}\frac{d^2\psi}{dx^2}-\bar V(x)\psi\right)dx$$
$$=\int_{-\infty}^\infty\psi^*\left(-\frac{1}{2m}\frac{d^2\psi}{dx^2}\right)dx+\int_{x_1}^{x_2}-v_0\psi dx$$ 
$$=\frac{\alpha}{4m}-\frac{v_0}{2}(\mathrm{erf}(x_2\sqrt{\alpha})-\mathrm{erf}(x_1\sqrt{\alpha}))$$ 
Taking the partial derivative with respect to $\alpha$ 
$$\frac{\partial}{\partial\alpha}\bra\psi_\alpha\bar H\ket\psi_\alpha=\frac{1}{4m}-\frac{v_0}{2\sqrt{\pi\alpha}}(x_2e^{-\alpha x_2^2}-x_1e^{-\alpha x_1^2})$$
Note that the partial derivative is continuous for $x>0$. Next, take the limit as $\alpha\to 0$ where $x_2>x_1$.
$$\lim_{\alpha\to 0}\bra\psi_\alpha\bar H\ket\psi_\alpha=0$$
$$\lim_{\alpha\to 0}\frac{\partial}{\partial\alpha}\bra\psi_\alpha\bar H\ket\psi_\alpha=-\infty$$ 
Using Lemma 4, $\exists\alpha>0: \bra\psi_a\bar H\ket\psi_a<0$.
$$\because E_0\leq\bra\psi\hat H\ket\psi\leq\bra\psi\bar H\ket\psi<0$$ 
For all attractive potentials in one dimension there must exist at least one bound state.
\end{proof}
\end{sol}