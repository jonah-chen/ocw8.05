\begin{sol}
\\\textbf{Define}
$$\cos^2\phi\equiv|\braket{\Psi,0}{\Psi,t}|^2,\phi\in\left[0,\frac{\pi}{2}\right]$$
$$Q\equiv\ketbra{\Psi,0}{\Psi,0}$$
\begin{enumerate}[label=\textbf{(\alph*)}]
\item
First, note the operator $Q$ is time-independent. 
$$\frac{d}{dt}\cos^2\phi=\frac{d}{dt}|\braket{\Psi,0}{\Psi,t}|^2$$
$$-2\cos\phi\sin\phi\frac{d\phi}{dt}=\frac{d}{dt}\left(\braket{\Psi,t}{\Psi,0}\braket{\Psi,0}{\Psi,t}\right)$$
$$-\sin(2\phi)\frac{d\phi}{dt}=\frac{d}{dt}\bra{\Psi,t}Q\ket{\Psi,t}$$
$$-\sin(2\phi)\frac{d\phi}{dt}=\frac{d\langle Q\rangle}{dt}$$
It is known that $\sin(2\phi)\in[0,1]\forall\phi\in\left[0,\frac{\pi}{2}\right]$, thus, it can be established that
$$\bigg|\frac{d\phi}{dt}\bigg|\leq\bigg|\frac{d\langle Q\rangle}{dt}\bigg|$$
From the uncertainty principle,
$$\Delta E\Delta Q\geq\frac{1}{2}\bigg|\frac{d\langle Q\rangle}{dt}\bigg|$$
Substituting for $\big|\frac{d\langle Q\rangle}{dt}\big|$ into the inequality,
$$\bigg|\frac{d\phi}{dt}\bigg|\leq 2\Delta E\Delta Q$$
Since the operator $Q$ is known to be a projector, the uncertainty is defined as 
$$(\Delta Q)^2=\langle Q^2\rangle-\langle Q\rangle^2=\langle Q\rangle-\langle Q\rangle^2$$
It is known that $\langle Q\rangle\in[0,1]$ for any states. Since $(\Delta Q)^2$ is quadratic in $\langle Q\rangle$, it is trivial to find the minimum value which is $(\Delta Q)^2=\frac{1}{4}$ Thus,
$$\bigg|\frac{d\phi}{dt}\bigg|\leq \Delta E$$
\begin{lemma}
If the system is governed by a time-independent Hamiltonian, the expectation value for all positive powers of the Hamiltonian must be time-independent.
\end{lemma}
\begin{proof}
Since the Hamiltonian is a hermitian operator, all states can be written as a superposition of eigenstates. Let $\ket{\Psi,0}$ be an arbitrary normalized state at a initial time and $\ket{E}$ be the orthonormal eigenstates of the Hamiltonian with eigenvalue $E$
$$\ket{\Psi,0}=\int\ketbra{E}{E}\ket{\Psi,0} dE$$
At the initial time, the expectation value of $\hat H^n$ is
$$\langle \hat H^n\rangle(0)=\braket{\Psi,0}{\Psi,0}=\int\bra{E}\hat H^n\ketbra{E}{E}\ket{\Psi,0}dE=\int E^n\ketbra{E}{E}\ket{\Psi,0}dE$$
Since the Hamiltonian is time independent, its eigenstates evolve with a phase $e^{-iEt}$, thus
$$\ket{\Psi,t}=\int e^{-iEt}\ketbra{E}{E}\ket{\Psi,0} dE$$
The expectation value of $\hat H^n$ is
$$\langle \hat H^n\rangle(t)=\braket{\Psi,t}{\Psi,t}=\int\bra{E}e^{iEt}\hat H^ne^{-iEt}\ketbra{E}{E}\ket{\Psi,0}dE$$$$=\int E^n\ketbra{E}{E}\ket{\Psi,0}dE=\langle \hat H^n\rangle(0)$$
\end{proof}
The minimum time it will take for the system to evolve into an orthogonal state can be computed with
$$\int_0^{t_{min}}\frac{d\phi}{dt}dt=\frac{\pi}{2}$$
As the minimum time is concerned, the maximum value of $\frac{d\phi}{dt}$ can be used. Due to Lemma 11, the maximum value is $\Delta E$, resulting in
$$\int_0^{t_{min}}\Delta Edt=\frac{\pi}{2}$$
$$t_{min}\Delta E=\frac{\pi}{2}$$
Since $t_{min}$ is the minimum time required, a general inequality can be constructed to be
$$\Delta E\Delta t_\perp\geq\frac{\pi}{2}$$

\item
 $\phi(0)$ is defined to be $0$. It has been shown that 
$$\bigg|\frac{d\phi}{dt}\bigg|\leq \Delta E$$
Define $\phi_{min}(t_1),\phi_{max}(t_1):\phi_{min}(t)\leq\phi(t)\leq\phi_{max}(t)\forall t\in[0,t_1]$  
$$\phi(t)_{min}\geq\phi(0)-\int_0^t\bigg|\frac{d\phi}{dt}\bigg|_{max}dt=\phi(0)-\Delta Et=-\Delta Et$$
$$\phi(t)_{max}\leq\phi(0)+\int_0^t\bigg|\frac{d\phi}{dt}\bigg|_{max}dt=\phi(0)+\Delta Et=\Delta Et
$$

Let $\displaystyle{g\equiv\frac{\pi}{2\Delta E}}$. It is known that $\displaystyle{\Delta E\Delta t_\perp\geq\frac{\pi}{2}}$ or $\Delta t_\perp\geq g$.
$$\therefore \phi(\Delta t_\perp)_{min}\geq-\frac{\pi}{2},\phi(\Delta t_\perp)_{max}\leq\frac{\pi}{2}$$
Within the interval $[0,\Delta t_\perp]$, $\phi$ is constrained within the interval $\left[-\frac{\pi}{2},\frac{\pi}{2}\right]$. Thus, $\cos^2(\phi)$ is constrained within the interval $[0,1]$. 
$$\because\cos^2(\theta)=\cos^2(-\theta), \forall t\in[0,\Delta t_\perp], \cos^2(\phi)\geq\cos^2(\Delta Et)$$ $$\therefore|\braket{\Psi,0}{\Psi,t}|^2\geq\cos^2(\Delta Et)$$ 


\end{enumerate}
\end{sol}