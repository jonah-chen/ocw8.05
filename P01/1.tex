\begin{sol}
\begin{enumerate}[label=\textbf{(\alph*)}]
	\item The normalization condition requires $\braket{\psi}{\psi}=1$.
	\begin{equation}
		\begin{aligned}
			\braket{\psi}{\psi}&=\int_{-\infty}^{\infty}\psi^*(x)\psi(x) dx\\
			&=N^2\int_{-\infty}^{\infty}x^2\exp(-\alpha x^2)dx=\frac{N^2\sqrt\pi}{2\alpha^\frac{3}{2}}=1\\
		\end{aligned}
	\end{equation}
	\begin{equation}
		N=\sqrt{\frac{2\alpha^\frac{3}{2}}{\sqrt{\pi}}}
	\end{equation}
	
	
	The expected value of $x$ position is:
	\begin{equation}
		\langle\hat{x}\rangle=N^2\int_{-\infty}^{\infty}x^3\exp(-\alpha x^2)dx=0
	\end{equation}
	Similarly, the expected value of $x^2$ position is:
	\begin{equation}
		\langle\hat{x^2}\rangle=N^2\int_{-\infty}^{\infty}x^4\exp(-\alpha x^2)dx=\frac{2\alpha^\frac{3}{2}}{\sqrt{\pi}}\frac{3\sqrt{\pi}}{4\alpha^\frac{5}{2}}=\frac{3}{2\alpha}
	\end{equation}
	
	\item The momentum operators $\hat{p}$ and $\hat{p^2}$ are given as:
	\begin{align}
		\hat{p}&=-i\partial_x\\
		\hat{p^2}&=-\partial_x^2
	\end{align}
	These operators acting on the state $\psi$ gives the following:
	\begin{align} 
		\hat{p}\psi&=-i\partial_x\left(Nx\exp(-\frac{1}{2}\alpha x^2)\right)=
	iN(\alpha x^2-1)\exp(-\frac{1}{2}\alpha x^2)\\
	\hat{p^2}\psi&=-\partial_x^2\left(Nx\exp(-\frac{1}{2}\alpha x^2)\right)=
	N(3\alpha x - \alpha^2x^3)\exp(-\frac{1}{2}\alpha x^2)
	\end{align}
	The expected value of momentum is $\langle\hat{p}\rangle$.
	\begin{equation}
		\langle\hat{p}\rangle=iN^2\int_{-\infty}^{\infty}(\alpha x^3-x)\exp(-\alpha x^2)dx=0
	\end{equation}
	Similarly, the expected value of $p^2$ is $\langle\hat{p^2}\rangle$.
	\begin{equation}
		\begin{aligned}
			\langle\hat{p^2}\rangle&=N^2\int_{-\infty}^{\infty}3\alpha x^2 \exp(-\alpha x^2)dx-
		\int_{-\infty}^{\infty}\alpha^2 x^4 \exp(-\alpha x^2)dx\\&=\frac{2\alpha^\frac{3}{2}}{\sqrt{\pi}}
		\left(\frac{3\sqrt\pi}{2\sqrt\alpha}-\frac{3\sqrt{\pi}}{4\sqrt{\alpha}}\right)
		=\frac{3\alpha}{2}
		\end{aligned}
	\end{equation}
	
	\item It is previously shown that $\langle \hat{x^2} \rangle > \langle \hat{x}\rangle^2$ and $\langle \hat{p^2} \rangle > \langle \hat{p}\rangle^2$, hence, the uncertainties for both momentum and position are non-zero. Therefore, the state $\ket{\psi}$ is neither a position eigenstate nor a momentum eigenstate.
	\item Since $\hat{H}=\frac{\hat{p^2}}{2m}+V$ and $V=0$. 
	\begin{equation}
		\langle\hat{H}\rangle=\frac{\langle\hat{p^2}\rangle}{2m}=\frac{3\alpha}{4m}
	\end{equation}
	\item If the state described by $\psi$ is an energy eigenstate, it must satisfy the time-independent Schrodinger's equation.
	\begin{align}
		2m(V(x)-E)\ket\psi&=\partial_x^2\ket\psi\\
		2m(V(x)-E)Nx\exp(-\frac{1}{2}\alpha x^2)&=N(\alpha^2x^3-3\alpha x)\exp(-\frac{1}{2}\alpha x^2)\\
		2m(V(x)-E)&=\alpha^2x^2-3\alpha
	\end{align}

	Given the additional condition that $V(0)=0$, $E=\displaystyle{\frac{3\alpha}{2m}}$.
	\\
	\begin{align}
		2m(V(x))-3\alpha&=\alpha^2x^2-3\alpha\\
		V(x)&=\frac{\alpha^2x^2}{2m}
	\end{align}	
	
	Since $\psi$ has a node at $x=0$, the node theorem prohibits $\psi$ to be the ground state of the potential.
	
\end{enumerate}
\end{sol}