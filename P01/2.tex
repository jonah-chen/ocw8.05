\begin{sol}
	Given a single valued potential $V(x)$: $\mathbb{R}\to\mathbb{R}$ defined within the infinate interval and $V(x)>V_0$ for all $x \in\mathbb{R}$. \\
	An stationary state must satisfy the time-independent Schrodinger's equation
	$$2m(V(x)-E)\psi=\psi''$$
	Assume there exists a well-behaved ground state $\ket\psi$ with an energy eigenvalue $E|E<V_0$. For this state $\ket\psi$, $V(x)-E>0$ for all $x \in\mathbb{R}$. Without loss of generality, it can be assumed that $\psi(x)$ is real. Hence, a position $x_0$ can be chosen such $\psi(x_0)$ and $\psi'(x_0)$ are both nonzero unless $\psi(x)=0$, which is not normalizable. If $\psi(x_0)\psi'(x_0)>0$, then $\psi'(x_0)\psi''(x_0)>0$, causing $\psi(x)$ to diverge in the positive $x$ direction which makes the state not normalizable. If $\psi(x_0)\psi'(x_0)<0$, then $\psi'(x_0)\psi''(x_0)<0$ which cause $\psi(x)$ to reach zero in the positive $x$ direction, resulting in the violation of the node theorem. If it is impossible to construct the ground state with $E<V_0$, it is impossible to construct any physical state with $E<V_0$ since the ground state is the lowest energy state by definition. 
\end{sol}