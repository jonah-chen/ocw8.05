\begin{sol}
\begin{enumerate}[label=\textbf{(\alph*)}]
    \item
    The probability density and probability current density in one dimension are defined as follows:
    $$\rho(x,t)=\Psi^*\Psi,J(x,t)=\frac{1}{m}Im\left(\Psi^*\frac{\partial\Psi}{\partial x}\right).$$
    Notice $Im(z)=\frac{1}{2i}(z-z^*)$. Thus, the probability current can be rewritten as:
    $$J(x,t)=\frac{1}{2mi}\left(\Psi^*\frac{\partial\Psi}{\partial x}-\Psi\frac{\partial\Psi^*}{\partial x}\right)$$
    The following partial derivatives can be computed.
    $$\frac{\partial\rho}{\partial t}=\Psi^*\frac{\partial\Psi}{\partial t}+\Psi\frac{\partial\Psi^*}{\partial t}$$ $$\frac{\partial J}{\partial x}=\frac{1}{2mi}\left(\Psi^*\frac{\partial^2\Psi}{\partial x^2}-\Psi\frac{\partial^2\Psi^*}{\partial x^2}\right)$$ 
    Assuming the potential is real and time-independent, the time-dependent Schrodinger's equation can be written as:
    $$\frac{\partial^2\Psi}{\partial x^2}=-2mi\frac{\partial\Psi}{\partial t}+V(x)\Psi$$ 
    The complex conjugate of the time-dependent Schrodinger's equation is then
     $$\frac{\partial^2\Psi^*}{\partial x^2}=2mi\frac{\partial\Psi^*}{\partial t}+V(x)\Psi^*$$
    Substituting the second partial derivatives into the equation for $\frac{\partial J}{\partial x}$
    $$\frac{\partial J}{\partial x}=\frac{1}{2mi}\left(\Psi^*\left(-2mi\frac{\partial\Psi}{\partial t}+V(x)\Psi\right)-\Psi\left(2mi\frac{\partial\Psi^*}{\partial t}+V(x)\Psi^*\right)\right)$$
    $$=-\left(\Psi^*\frac{\partial\Psi}{\partial t}+\Psi\frac{\partial\Psi^*}{\partial t}\right)$$
    From this form, it is trivial to see that
    $$\frac{\partial\rho}{\partial t}+\frac{\partial J}{\partial x}=0$$ 
    The dimensions of $\rho$ is $Length^{-1}$, and the dimensions of $J$ is $Time^{-1}$.
   \item
   $$\frac{dP_{ab}}{dt}=\frac{d}{dt}\int_{a}^{b}\rho(x,t)dx$$ 
   Given $\rho(x,t)$ is well behaved in the region $a<x<b$, differentiation under the integral sign can be performed.
   $$\frac{dP_{ab}}{dt}=\int_{a}^{b}\frac{\partial\rho}{\partial t}dx=
   \int_{a}^{b}-\frac{\partial J}{\partial x}dx=J(a,t)-J(b,t)$$ If a state is normalized, then $\lim_{x\to\pm\infty}\Psi(x,t)=0$ and the probability $\lim_{a\to-\infty}\lim_{b\to\infty}P_{ab}=1$. It has been shown that $\frac{dP_{ab}}{dt}=J(a,t)-J(b,t)$. To show that the total if a state $\Psi(x,t)$ that is normalized at time $t$ will remain normalized, it suffices to show that $\lim_{x\to\pm\infty}J(x,t)=0$. \\
   Recall $$J(x,t)=\frac{1}{m}\text{Im}\left(\Psi^*\frac{\partial\Psi}{\partial x}\right)$$
   Under the assumption that $\frac{\partial\Psi}{\partial x}$ does not diverge at $\pm\infty$, it is trivial that $\lim_{x\to\pm\infty}J(x,t)=0$ since $\lim_{x\to\pm\infty}\Psi(x,t)=0$. Thus, a state that is normalized at time $t$ will remained normalized for all times.
   \item 
   For $\psi(x)=e^{i\alpha(x)}\phi(x)$
  
   $$=\frac{1}{m}\text{Im}((e^{-i\alpha(x)}\phi(x))(e^{i\alpha(x)}\phi'(x)+i\alpha'(x)e^{i\alpha(x)}\phi(x))$$
$$=\frac{1}{m}\alpha'(x)\phi(x)^2$$ The probability density $\rho(x,t)=\phi(x)^2$. Thus, it is shown that
$$\frac{J(x)}{\rho(x)}=\frac{1}{m}\alpha'(x)$$ 
Notice this quantity has dimensions of velocity. Furthermore, the current is normalized is against the probability density at a point, thus, this ratio can be viewed as the local velocity of a quantum state.
\item
For $\psi(x)=Ae^{ipx}+Be^{-ipx}$
$$J(x,t)=\frac{1}{m}\text{Im}((A^*e^{-ipx}+B^*e^{ipx})ip(Ae^{ipx}-Be^{-ipx}))$$
$$=\frac{1}{m}\text{Im}(ip(|A|^2-|B|^2+AB^*e^{2ipx}-A^*Be^{-2ipx}))$$
$$=\frac{p}{m}(|A|^2-|B|^2+Re(AB^*e^{2ipx}-A^*Be^{-2ipx}))$$ 
Yes, there are cross terms in $J$ between the left and right moving parts of $\psi$ if $A\neq B$.

\end{enumerate}
\end{sol}