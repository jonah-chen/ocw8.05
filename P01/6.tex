\begin{sol}
\begin{enumerate}[label=\textbf{(\alph*)}]
    \item
    The probability density and probability current density in one dimension are defined as follows:
    \begin{align}
        \rho(x,t)&=\Psi^*\Psi\\
        J(x,t)&=\frac{1}{m}\text{Im}\left(\Psi^*\frac{\partial\Psi}{\partial x}\right)\label{probabilitycurrent}
    \end{align}
    Notice $\text{Im}(z)=\frac{1}{2i}(z-z^*)$. Thus, the probability current can be rewritten as:
    \begin{equation}
        J(x,t)=\frac{1}{2mi}\left(\Psi^*\frac{\partial\Psi}{\partial x}-\Psi\frac{\partial\Psi^*}{\partial x}\right)
    \end{equation}
    The following partial derivatives can be computed.
    \begin{align}
        \partial_t\rho&=\Psi^*\partial_t\psi+\Psi\partial_t\Psi^*\\
        \partial_xJ&=\frac{1}{2mi}\left(\Psi^*\partial_x^2\Psi-\Psi\partial_x^2\Psi^*\right)
    \end{align}

    Assuming the potential is real and time-independent, the time-dependent Schrodinger's equation can be written as:
    \begin{equation}
        \partial_x^2\Psi=-2mi\partial_t\Psi+V(x)\Psi 
    \end{equation}
    The complex conjugate of the time-dependent Schrodinger's equation is then
    \begin{equation}
        \partial_x^2\Psi^*=2mi\partial_t\Psi^*+V(x)\Psi^*
    \end{equation}
    Substituting the second partial derivatives into the equation for $\partial_xJ$
    \begin{equation}
        \begin{aligned}
            \partial_xJ&=\frac{1}{2mi}\left(\Psi^*\left(-2mi\partial_t\Psi+V(x)\Psi\right)-\Psi\left(2mi\partial_t\Psi^*+V(x)\Psi^*\right)\right)\\
            &=-\left(\Psi^*\partial_t\Psi+\Psi\partial_t\Psi^*\right)
        \end{aligned}
    \end{equation}
    From this form, it is trivial to see that
    \begin{equation}
        \partial_t\rho+\partial_xJ=0
    \end{equation}
    This is the conservation of probability. The dimensions of $\rho$ is $Length^{-1}$, and the dimensions of $J$ is $Time^{-1}$.
   \item
   \begin{equation}
       \partial_tP_{ab}=\partial_t\int_{a}^{b}\rho(x,t)dx
   \end{equation}
   Given $\rho(x,t)$ is well behaved in the region $a<x<b$, differentiation under the integral sign can be performed.
   \begin{equation}
    \partial_tP_{ab}=\int_{a}^{b}\partial_t\rho dx=
    \int_{a}^{b}-\partial_tJdx=J(a,t)-J(b,t)
   \end{equation}
   If a state is normalized, then $\lim_{x\to\pm\infty}\Psi(x,t)=0$ and the probability $\lim_{a\to-\infty}\lim_{b\to\infty}P_{ab}=1$. It has been shown that $\frac{dP_{ab}}{dt}=J(a,t)-J(b,t)$. To show that the total if a state $\Psi(x,t)$ that is normalized at time $t$ will remain normalized, it suffices to show that $\lim_{x\to\pm\infty}J(x,t)=0$. \\
   Recall equation \eqref{probabilitycurrent}. Under the assumption that $\partial_x\Psi$ does not diverge at $\pm\infty$, it is trivial that $\lim_{x\to\pm\infty}J(x,t)=0$ since $\lim_{x\to\pm\infty}\Psi(x,t)=0$. Thus, a state that is normalized at time $t$ will remained normalized for all times.
   \item 
   For $\psi(x)=e^{i\alpha(x)}\phi(x)$
   \begin{equation}
        J(x,t)=\frac{1}{m}\text{Im}((e^{-i\alpha(x)}\phi(x))(e^{i\alpha(x)}\phi'(x)+i\alpha'(x)e^{i\alpha(x)}\phi(x))=\frac{1}{m}\alpha'(x)\phi(x)^2
   \end{equation}
   The probability density $\rho(x,t)=\phi(x)^2$. Thus, it is shown that
   \begin{equation}
    \frac{J(x)}{\rho(x)}=\frac{1}{m}\alpha'(x)
   \end{equation}
Notice this quantity has dimensions of velocity. Furthermore, the current is normalized is against the probability density at a point, thus, this ratio can be viewed as the local velocity of a quantum state.
\item
For $\psi(x)=Ae^{ipx}+Be^{-ipx}$,
\begin{equation}
    \begin{aligned}
        J(x,t)&=\frac{1}{m}\text{Im}((A^*e^{-ipx}+B^*e^{ipx})ip(Ae^{ipx}-Be^{-ipx}))\\
        &=\frac{1}{m}\text{Im}(ip(|A|^2-|B|^2+AB^*e^{2ipx}-A^*Be^{-2ipx}))\\
        &=\frac{p}{m}(|A|^2-|B|^2+Re(AB^*e^{2ipx}-A^*Be^{-2ipx}))
    \end{aligned}
\end{equation}

Yes, there are cross terms in $J$ between the left and right moving parts of $\psi$ if $A\neq B$.

\end{enumerate}
\end{sol}