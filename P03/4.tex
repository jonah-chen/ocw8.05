\begin{sol}
\begin{lemma}
$\mathbf{a}\cdot\mathbf{b}=(\mathbf{b}\cdot\mathbf{a})^*$
\end{lemma}
\begin{proof} 
\begin{equation}
	\mathbf a^\dagger\mathbf b=\mathbf{a}_i^*\mathbf{b}_i
\end{equation}
\begin{equation}
	\mathbf b^\dagger\mathbf a=\mathbf{a}_i\mathbf{b}_i^*=(\mathbf a^*_i\mathbf b_i)^*
\end{equation}
Note: for $\mathbf a, \mathbf b\in\mathbb{R}^n$, $\mathbf{a}\cdot\mathbf{b}=\mathbf{b}\cdot\mathbf{a}$.
\end{proof}
\begin{enumerate}[label=\textbf{(\alph*)}]
\item
\begin{equation}
	f(\lambda)\equiv|\mathbf{a}-\lambda\mathbf{b}|^2\geq 0
\end{equation} 
\begin{equation}
	=(\mathbf{a}-\lambda\mathbf{b})\cdot(\mathbf{a}-\lambda\mathbf{b})=(\mathbf{a}-\lambda\mathbf{b})^\dagger(\mathbf{a}-\lambda\mathbf{b})=(\mathbf{a}^\dagger-\lambda\mathbf{b}^\dagger)(\mathbf{a}-\lambda\mathbf{b})
\end{equation} 
\begin{equation}
	=\mathbf{a}^\dagger\mathbf a+\lambda^2\mathbf b^\dagger \mathbf b-\lambda (\mathbf a^\dagger\mathbf b+\mathbf{ab}^\dagger)
\end{equation}
Using Lemma 6,
\begin{equation}
	f(\lambda)=|\mathbf{a}|^2+\lambda^2|\mathbf{b}|^2-2\lambda (\mathbf a\cdot\mathbf b)
\end{equation}
Since $f$ is a quadratic function in $\lambda$ and $|\mathbf{a}|^2$ and $|\mathbf{b}|^2$ are strictly non-negative, the minimum point for $f$ is at $\lambda_{min}=\frac{\mathbf{a}\cdot\mathbf{b}}{|\mathbf{b}|^2}$. \begin{equation}
	\because f(\lambda)\geq 0\forall x\in\mathbb{R}, f(\lambda_{min})\geq 0
\end{equation} 
\begin{equation}
	\min_{\lambda}f(\lambda)\equiv f(\lambda_{min})=|\mathbf{a}|^2+\frac{(\mathbf{a}\cdot\mathbf{b})^2}{|\mathbf{b}|^2}-\frac{2(\mathbf{a}\cdot\mathbf{b})^2}{|\mathbf{b}|^2}\geq 0
\end{equation}
\begin{equation}
	|\mathbf{a}|^2|\mathbf{b}|^2-(\mathbf{a}\cdot\mathbf{b})^2\geq 0
\end{equation}
\begin{equation}
	|\mathbf{a}\cdot\mathbf{b}|^2\leq|\mathbf{a}||\mathbf{b}|
\end{equation} 
The Schwarz inequality is saturated when $\lambda=\lambda_{min}$ or when $\displaystyle{\mathbf{b}=\frac{\mathbf{a}\cdot\mathbf{b}}{|\mathbf b|^2}\mathbf{a}}$.
\begin{lemma}
Given $f(z):\mathbb{C}\to\mathbb{C}$ is a holomorphic function for $z\in\mathbb{C}$, $Re(z_0)\neq 0, Im(z_0)\neq 0$
\begin{equation}
	\frac{\partial f}{\partial z_0}(w)=0 \text{ and }\frac{\partial f}{\partial z_0^*}(w)=0\implies w \text{ is a stationary point of }f
\end{equation}
\end{lemma}
\begin{proof}
A stationary point of $f$ is defined to be a point $z$ where
\begin{equation}
	\lim_{\delta z\to 0,\delta z\in\mathbb{C}} \frac{f(w+\delta z)-f(z)}{\delta w}=0
\end{equation}
Let $\epsilon$ be a small real number and $u\in\mathbb C$. The equation above is equivalent to 
\begin{equation}
	\lim_{\epsilon\to 0} \frac{f(w+u\epsilon)-f(w)}{u\epsilon}=0
\end{equation} 
This limit is also the definition of $\displaystyle{\frac{\partial f}{\partial u}}$. 
\\
\begin{equation}
	\text{Re}(z_0)\neq 0, \text{Im}(z_0)\neq 0\implies\forall z\in\mathbb{C}=\alpha z_0+\beta z_0^*, (\alpha,\beta)\in\mathbb{R}^2
\end{equation}
\begin{equation}
	\therefore\frac{\partial f}{\partial u}=\alpha\frac{\partial f}{\partial z_0}+\beta\frac{\partial f}{\partial z_0^*}=0
\end{equation} 
Define $w=u\epsilon$ and the defining condition for a stationary point of $f$ is satisfied. 
\end{proof}
\item
\begin{equation}
	f(\lambda)\equiv\braket{\nu(\lambda)}{\nu(\lambda)}\geq 0
\end{equation}
\begin{equation}
	=\braket{\ket a-\lambda\ket b}{\ket a-\lambda\ket b}=\braket{a}{a}+|\lambda|^2\braket{b}{b}-\lambda\braket{a}{b}-\lambda^*\braket{b}{a})
\end{equation} 
Using Lemma 6, 
\begin{equation}
	f(\lambda)=|a|^2+|\lambda|^2|b|^2-2\lambda\braket{a}{b}=|a|^2+\lambda^*\lambda|b|^2-\lambda\braket{a}{b}+\lambda^*\braket{a}{b}^*
\end{equation}
Since $f$ is a quadratic function in $\lambda$ and $\lambda^*$ and $\braket{a}{a}$ and $\braket{b}{b}$ are non-negative, the stationary point of $f$ is a global minimum, which can be found using Lemma 7.
\begin{equation}
	\frac{\partial f}{\partial\lambda}=\lambda^*|b|^2-\braket{a}{b}=0
\end{equation} 
\begin{equation}
	\frac{\partial f}{\partial\lambda^*}=\lambda|b|^2-\braket{a}{b}^*=0
\end{equation} 
\begin{equation}
	\therefore\lambda_{min}=\frac{\braket{a}{b}^*}{|b|^2},\lambda^*_{min}=\frac{\braket{a}{b}}{|b|^2}
\end{equation}
Note that $\lambda_{min}^*\lambda_{min}|b|^4=|\lambda_{min}|^2|b|^4=|\braket{a}{b}|^2$.
\begin{equation}
	f(\lambda_{min})=|a|^2+\frac{|\braket{a}{b}|^2}{|b|^2}-\frac{|\braket{a}{b}|^2}{|b|^2}-\frac{|\braket{a}{b}|^2}{|b|^2}=|a|^2-\frac{|\braket{a}{b}|^2}{|b|^2}\geq 0
\end{equation}
\begin{equation}
	|\braket{a}{b}|^2\leq\braket{a}{a}\braket{b}{b}
\end{equation} 
\item
The \textit{triangle inequality} is $|a+b|\leq|a|+|b|$.\\
Squaring both sides of the inequality results in:
\begin{equation}
	|a+b|^2=\braket{a+b}{a+b}=\braket{a}{a}+\braket{b}{b}+\braket{a}{b}+(\braket{a}{b})^*
\end{equation}
\begin{equation}
	=\braket{a}{a}+\braket{b}{b}+2\text{Re}(\braket{a}{b})
\end{equation} 
\begin{equation}
	(|a|+|b|)^2=\braket{a}{a}+\braket{b}{b}+2\braket{a}{b}
\end{equation}
\begin{equation}
	\because\forall z\in\mathbb C, |\text{Re}(z)|\leq|z|, |a+b|^2\leq(|a|+|b|)^2
\end{equation} 
Then, the \textit{triangle inequality} will miraculously manifest from taking the square root of both sides.
\end{enumerate}
\end{sol}