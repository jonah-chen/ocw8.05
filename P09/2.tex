\begin{sol}
        \begin{enumerate}[label=\textbf{(\alph*)}]
            \item
            The rotation operator can be expanded into its series
            \begin{align}
                R_\epsilon(\mathbf{n})&=\exp\left(- i\epsilon\mathbf{n}\cdot\mathbf{S} \right)=1-i\epsilon\mathbf{n}\cdot\mathbf{S}+\mathcal{O}\left( \epsilon^2 \right)\\
                R_\epsilon^\dagger(\mathbf{n})&=\exp\left( -i\epsilon\mathbf{n}\cdot\mathbf{S} \right)=1+i\epsilon\mathbf{n}\cdot\mathbf{S}+\mathcal{O}\left( \epsilon^2 \right)\\
                R_\epsilon^\dagger(\mathbf{n})\mathbf{S}R_\epsilon^\dagger(\mathbf{n})&=\mathbf{S}+i\epsilon\mathbf{n\cdot S}\mathbf{S}-i\epsilon\mathbf{Sn\cdot S}=\mathbf{S}+i\epsilon[\mathbf{n\cdot S},\mathbf{S}]=\mathbf{S}+\epsilon(\mathbf{n}\times\mathbf{S})
            \end{align}
        \item
            The expectation value of the rotated state obtained by the rotation operator (up to first order in $\epsilon$) acting on the unrotated spin state $\ket{\mathbf{n'};+}$ yields the two terms.
            \begin{equation}
                \bra{\mathbf{n'};+}R_\epsilon^\dagger(\mathbf{n})\mathbf{S}R_\epsilon(\mathbf{n})\ket{\mathbf{n'};+} = \bra{\mathbf{n'};+}\mathbf{S}\ket{\mathbf{n'};+}+\epsilon\bra{\mathbf{n'};+}\mathbf{n}\times\mathbf{S}\ket{\mathbf{n'};+}
            \end{equation}
            Note that the first term represents the expectation value of spin for the unrotated spin state $\ket{\mathbf{n'};+}$ in the same direction as $\mathbf S$, and the second term represent a small contribution of the expectation value for the same state in the perpindicular direction as $\mathbf{S}$, as denoted by the cross product. Thus, the rotation operator does indeed rotate the spin state about the normal vector $\mathbf{n}$.
        \end{enumerate}
    \end{sol}    