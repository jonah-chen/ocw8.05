\begin{sol}
The expectation value of the spin operator on the arbitrary spin state can be calculated separately. As the arbitrary spin state and the spin operators are known in the orthonormal z basis
    \begin{equation}
        \ket{n;+}=\begin{pmatrix} \cos\left( \frac{\theta}{2} \right) \\e^{i\phi}\sin\left( \frac{\theta}{2} \right)  \end{pmatrix} 
    \end{equation}
    \begin{equation}
        \begin{aligned}
            \left<\hat{S}_x \right> &= \bra{n;+}\hat{S}_x\ket{n;+}=\frac{1}{2}\begin{pmatrix} \cos\left( \frac{\theta}{2} \right)&e^{-i\phi}\sin\left( \frac{\theta}{2} \right)   \end{pmatrix} 
            \begin{pmatrix} 0&1\\1&0 \end{pmatrix}
            \begin{pmatrix} \cos\left( \frac{\theta}{2} \right)\\e^{i\phi}\sin\left( \frac{\theta}{2} \right)   \end{pmatrix}\\
            &=\frac{1}{2}\cos\left( \frac{\theta}{2} \right)\sin\left( \frac{\theta}{2} \right)\left(e^{i\phi}+e^{-i\phi}\right)=\frac{1}{2}\sin\theta\cos\phi  
        \end{aligned}
    \end{equation}
    \begin{equation}
        \begin{aligned}
            \left<\hat{S}_y \right> &= \bra{n;+}\hat{S}_y\ket{n;+}=\frac{1}{2}\begin{pmatrix} \cos\left( \frac{\theta}{2} \right) & e^{-i\phi}\sin\left( \frac{\theta}{2} \right)  \end{pmatrix}
            \begin{pmatrix} 0&-i\\i&0 \end{pmatrix}\begin{pmatrix} \cos\left( \frac{\theta}{2} \right) \\ e^{i\phi}\sin\left( \frac{\theta}{2} \right)  \end{pmatrix}\\
            &=-\frac{i}{2}\cos\left( \frac{\theta}{2} \right)\sin\left( \frac{\theta}{2} \right)\left( e^{i\phi}-e^{-i\phi} \right)=\frac{1}{2}\sin\theta\sin\phi
        \end{aligned}
    \end{equation}
    \begin{equation}
        \begin{aligned}
            \left<\hat{S}_z \right> &= \bra{n;+}\hat{S}_z\ket{n;+}=\frac{1}{2}\begin{pmatrix} \cos\left( \frac{\theta}{2} \right)& e^{-i\phi}\sin\left( \frac{\theta}{2} \right)  \end{pmatrix} 
            \begin{pmatrix} 1&0\\0&-1 \end{pmatrix}\begin{pmatrix} \cos\left( \frac{\theta}{2} \right)\\e^{i\phi}\sin\left( \frac{\theta}{2} \right)   \end{pmatrix}\\
            &=\frac{1}{2}\left(\cos^2\left( \frac{\theta}{2} \right)-\sin^2\left( \frac{\theta}{2} \right)\right)=\frac{1}{2}\cos\theta  
        \end{aligned}
    \end{equation}  
    The vector expectation value can be written as
    \begin{equation}
        \left<\mathbf{S} \right>_\mathbf{n}\equiv\begin{pmatrix} \left<S_x \right>\\\left<S_y \right>\\\left<S_z \right> \end{pmatrix} =\frac{1}{2}\begin{pmatrix} \sin\theta\cos\phi\\ \sin\theta\sin\phi \\ \cos\theta \end{pmatrix}
    \end{equation}
    Given another unit vector $\mathbf{n'}$, the expectation value $\left<\mathbf{S}\cdot\mathbf{n'} \right>_\mathbf{n}$ can be computed by taking the dot product between the new normal vector and the spin expectation vector
    \begin{equation}
        \left<\mathbf{S}\cdot\mathbf{n'} \right>_\mathbf{n}\equiv \left<S_x \right>n'_x+\left<S_y \right>n'_y+\left<S_z \right>n'_z=\left<\mathbf{S} \right>_\mathbf{n}\cdot\mathbf{n'}=\frac{1}{2}\left(\sin\theta\cos\phi n'_x+\sin\theta\sin\phi n'_y+\cos\theta n'_z\right)
    \end{equation}
    In spherical coordinates, $\mathbf{n'}$ can be written as a function of the two angles $\theta'$ and $\phi'$.
    \begin{equation}
        \begin{aligned}
            \left<\mathbf{S}\cdot\mathbf{n'} \right>_\mathbf{n}&=\frac{1}{2}\left(\sin\theta\cos\phi\sin\theta'\cos\phi'+\sin\theta\sin\phi\sin\theta'\sin\phi'+\cos\theta\cos\theta'\right)\\
            &=\frac{1}{2}\left(\sin\theta\sin\theta'\cos\left(\phi-\phi' \right) + \cos\theta\cos\theta'\right)=\frac{1}{2}\mathbf{n}\cdot\mathbf{n'}
        \end{aligned}
    \end{equation}
\end{sol}