\begin{sol}
\begin{enumerate}[label=\textbf{(\alph*)}]
\item
Since the operator $\Omega$ has no explicit time dependence in the Schrodinger picture, the Heisenberg equation of motion is
$$\frac{d\Omega_H}{dt}=-i[\hat\Omega_H,\hat H_H]=-\frac{i}{2m}[\hat{\mathbf r}_H\cdot \hat{\mathbf p}_H,\hat{\mathbf p}_H^2]-i[\hat{\mathbf r}_H\cdot \hat{\mathbf p}_H,V(\hat {\mathbf{r}}_H)]$$  
Compute the first commutation relation by separating into components. The commutator $[\hat x\hat p,\hat p^2]$ is computed in \textit{4c, p.51} to be $2i\hat p^2$. It is also known that position and momentum along different axis commute.
$$[\hat{\mathbf r}_H\cdot \hat{\mathbf p}_H,\hat{\mathbf p}_H^2]=[r_xp_x,p_x^2]+[r_yp_y,p_y^2]+[r_zp_z,p_z^2]=2i\hat{\mathbf p}_H^2$$
The commutation relation $[\hat{\mathbf r}_H\cdot \hat{\mathbf p}_H,V(\hat {\mathbf{r}}_H)]$ is similar to that computed in \textit{4c, p.51}

$$[\hat{\mathbf r}_H\cdot \hat{\mathbf p}_H,V(\hat {\mathbf{r}}_H)]=\hat{\mathbf r}_H\cdot \hat{\mathbf p}_HV(\hat {\mathbf{r}}_H)-V(\hat {\mathbf{r}}_H)\hat{\mathbf r}_H\cdot \hat{\mathbf p}_H=-i\hat{ \mathbf r}_H\cdot(\nabla( V(\hat {\mathbf{r}}_H))-V(\hat {\mathbf{r}}_H)\nabla)$$ 
Using the chain rule
$$=-i\hat{ \mathbf r}_H\cdot(\nabla V(\hat {\mathbf{r}}_H)+V(\hat {\mathbf{r}}_H)\nabla-V(\hat {\mathbf{r}}_H)\nabla)=-i\hat{ \mathbf r}_H\cdot\nabla V(\hat {\mathbf{r}}_H)$$

$$\frac{d\Omega_H}{dt}=\frac{\hat{\mathbf p}_H^2}{m}-\hat {\mathbf r}_H\cdot\nabla V(\hat {\mathbf r}_H)$$ 
\item
The Heisenberg equation of motion holds for all Heisenberg operators, 
$$\frac{d\hat A_H}{dt}=\frac{\partial \hat A_S}{\partial t}-i[\hat A_H,\hat H_H]$$
If the Schrodinger operator is time-independent, the first term on the right hand side cancels. Expand the commutator to obtain
$$\frac{d\hat A_H}{dt}=i(\hat H_H\hat A_H-\hat A_H\hat H_H)$$ 
Take the expectation value of this operator on any stationary state $\ket{\Psi,0}$ with a energy eigenvalue $E$.
$$\left\langle\frac{d\hat A_H}{dt}\right\rangle=i(\bra{\Psi,0}\hat H_H\hat A_H\ket{\Psi,0}-\bra{\Psi,0}\hat A_H\hat H_H\ket{\Psi,0})$$$$=i(E\bra{\Psi,0}\hat A_H\ket{\Psi,0}-E\bra{\Psi,0}\hat A_H\ket{\Psi,0})=0$$
\item
For a stationary state of the Hamiltonian, the operator $\Omega$ must be time independent. This states that
$$\frac{d\Omega_H}{dt}=\frac{\hat{\mathbf p}_H^2}{m}-\hat {\mathbf r}_H\cdot\nabla V(\hat {\mathbf r}_H)=0$$ 
$$\therefore\frac{\hat{\mathbf p}_H^2}{m}=\hat {\mathbf r}_H\cdot\nabla V(\hat {\mathbf r}_H)$$
In spherical coordinates
$$\nabla V=\frac{\partial V}{\partial r}\hat r+\frac{1}{r\sin\phi}\frac{\partial V}{\partial\theta}\hat\theta+\frac{1}{r}\frac{\partial V}{\partial\phi}\hat\phi$$
Since a central potential is angularly independent, the gradient reduces to the first term. Similarly, the dot product reduces to a ordinary scalar product. For the potential $V(r)=c/r^k$,
$$\frac{\hat{\mathbf p}_H^2}{m}=r\frac{\partial V}{\partial r}=-r\frac{kc}{r^{k+1}}=-\frac{kc}{r^k}=-kV(r)$$  
Since the kinetic energy is equal to $\frac{\hat{\mathbf p}_H^2}{2m}$. Multiplying both sides by 2 and taking the expectation value yields
$$\langle T\rangle=-\frac{k}{2}\langle V\rangle$$

\end{enumerate}
\end{sol}