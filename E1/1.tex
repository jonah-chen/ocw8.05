\begin{sol}
\begin{enumerate}
    \item
    \textbf{False.} There exist only one energy eigenstate for $V_0<E<2V_0$ as the boundary condition for $x=-\infty$ will determine the unique energy eigenstate for each energy $V_0<E<2V_0$, since $V(x)=2V_0$ for $x<0$.
    \item
    \textbf{False.} There energy eigenstates for $0<E<V_0$ are bound states of the potential, thus, they must exist in a discrete spectrum.
    \item
    \textbf{True.} It has been shown that there cannot exist a well-behaved state with energy below the minimum energy of the potential, which is $E=0$
    \item
    \textbf{False.} The spin states pointing along the $x$-direction are $\ket{\pm_x}\frac{1}{\sqrt{2}}(\ket+\pm\ket-)$, therefore, the inner product between each of the states pointing along the $x$-direction are $|\braket{\pm_x}{+}|^2=\frac{1}{2}$.
    \item
    \textbf{False.} Since $S_x$ does not commute with $S_y$, $S_x^2$ cannot commute with $S_y^2$. 
    \item
    \textbf{True.} Let $A$ and $B$ be unitary operators. By definition, $A^\dagger A=B^\dagger B=\mathbf{1}$. Then, $(AB)^\dagger(AB)=B^\dagger A^\dagger AB=\mathbf{1}$. By definition, $AB$ is also a unitary operator.
    \item
    \textbf{False.} All eigenvectors of a projection operator must have eigenvalue of one.
    \item
    \textbf{False.} There exist operators that are not diagonlizable.Thus, two of these operators that commute cannot be diagonalized simultaneously.
    \item
    \textbf{False.} The exponential of an anti-hermitian operators is a unitary operator.
    \item
    \textbf{False.} The commutators of two hermitian operators is anti-hermitian.

\end{enumerate}
\end{sol}